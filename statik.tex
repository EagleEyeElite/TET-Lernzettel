\section{Statik}

Vereinfachte Maxwell-Gleichungen ($\frac{\partial}{\partial t} = 0$):
\begin{itemize}
\item $\text{div } \vec{D} = \rho$
\item $\text{rot } \vec{H} = \vec{J}$
\end{itemize}

Hier aufpassen: keine poission Gleichung lösen (da keine Ladung im Inneren des Raums ist, selbst Flächenladung braucht keine Poisson-Gleichung)

Randwertproblem/ Integral kommt auf jeden Fall als Rechenaufgabe in Klausur -> Also Potenzialproblem auf jeden Fall!

\subsection{Stetigkeitsbedingungen}

\begin{figure}[H]
\centering
\includesvg[width=0.45\textwidth]{1.svg}
\hfill
\includesvg[width=0.45\textwidth]{2.svg}
\end{figure}

\begin{itemize}
\item $D_{n2} - D_{n1} = \sigma$
\item $B_{n2} - B_{n1} = 0$
\item $E_{t2} - E_{t1} = 0$
\item $H_{t2} - H_{t1} = \vec{K}$
\end{itemize}

\subsection{Elektrostatik}

\subsubsection{Laplace-Operator Bedeutung}
\textbf{Physikalisch:}
\begin{align}
\Delta \Phi &= \nabla \cdot (\nabla \Phi) = \nabla \cdot \vec{E}
\end{align}
$\Delta \Phi = 0$ bedeutet: Keine Quellen/Senken, harmonischer Verlauf, ladungsfreies Gebiet


\textbf{Elektrostatik:}
\begin{align}
\text{Ladungsfrei:} \quad &\Delta \Phi = 0 \quad \text{(Laplace)}\\
\text{Mit Ladungen:} \quad &\Delta \Phi = -\frac{\rho}{\varepsilon} \quad \text{(Poisson)}\\
\text{Stationäre Strömung:} \quad &\text{div } \vec{J} = 0 \quad \text{(Potentialgleichung)}
\end{align}


\subsubsection{Laplace-Gleichung}
Allgemeine Form:
\begin{align}
\Delta\Phi(x, y, z) = \text{div grad } \Phi(x, y, z) = 0 \qquad (\text{div } \vec{D} = \varrho \text{, mit }\varrho = 0)
\end{align}




Kartesische Koordinaten $(x, y, z)$:
\begin{align}
\Delta \Phi &= \frac{\partial^2 \Phi}{\partial x^2} + \frac{\partial^2 \Phi}{\partial y^2} + \frac{\partial^2 \Phi}{\partial z^2} = 0
\end{align}
Zylinderkoordinaten $(\rho, \varphi, z)$:
\begin{align}
\Delta \Phi &= \frac{\partial^2 \Phi}{\partial \rho^2} + \frac{1}{\rho}\frac{\partial \Phi}{\partial \rho} + \frac{1}{\rho^2}\frac{\partial^2 \Phi}{\partial \varphi^2} + \frac{\partial^2 \Phi}{\partial z^2} = 0
\end{align}












\subsubsection{Separationsansatz}
Grundidee: Die Lösung einer partiellen Differentialgleichung wird als Produkt von Funktionen einzelner Variablen angesetzt:

\begin{align}
\Phi(x, y, z) &= f(x) \cdot g(y) \cdot h(z)\\
\Delta\Phi(x, y, z) &= \frac{d^2 f(x)}{dx^2} g(y)h(z) + f(x) \frac{d^2 g(y)}{dy^2} h(z) + f(x)g(y) \frac{d^2h(z)}{dz^2} = 0\\
% Division durch Φ(x,y,z)
&= \underbrace{\frac{1}{f} \frac{d^2 f}{dx^2}}_{K_x} +
\underbrace{\frac{1}{g} \frac{d^2 g}{dy^2}}_{K_y} +
\underbrace{\frac{1}{h} \frac{d^2h}{dz^2}}_{K_z} = 0
\end{align}

% Einzellösungen
Einzellösungen:
% Fall 1: Kx = 0
\begin{align}
f(x) &= \begin{Bmatrix} 1 \\ x \end{Bmatrix} \quad \text{mit } K_x = 0 \tag{3.43}\\
% Fall 2: Kx > 0
f(x) &= \begin{Bmatrix} e^{k_x x} \\ e^{-k_x x} \end{Bmatrix} \quad \text{mit } K_x = k_x^2 > 0 \tag{3.44}\\
% Fall 3: Kx < 0
f(x) &= \begin{Bmatrix} \sin(k_x x) \\ \cos(k_x x) \end{Bmatrix} \quad \text{mit } K_x = -k_x^2 < 0 \tag{3.45}
\end{align}


% Erklärung der Notation
Notation:
\begin{itemize}
    \item Die Konstanten $K$: Separationskonstante, als Quotient $f''/f$ definiert. 
    \item Die Konstanten $k$: Skalierungsfaktor (ähnlich wie später die Wellenzahl $k$). Man beachte, dass je nach Funktion $K_x = \pm k_x^2$ gilt.
    \item Die Separationsgleichung: $K_x + K_y + K_z = 0$.
\end{itemize}


\textbf{Flächenladung:}


Integration des Gaußschen Gesetzes:

\begin{align}
\int_V \nabla \cdot \vec{D} \, dV = \oint_{\partial V} \vec{D} \cdot \hat{n} \, dA = \int_V \rho \, dV = Q_{\text{eingeschlossen}}
\end{align}

Mantelbeiträge (Höhe $h \to 0$):
\begin{align}
D_y(y=0^+) \cdot dA - D_y(y=0^-) \cdot dA = \sigma(x,z) \cdot dA
\end{align}
\begin{align}
\Leftrightarrow D_y(y=0^+) - D_y(y=0^-) = \sigma(x,z)
\end{align}






















\subsubsection{Separationsansatz}

? (Bernoulli-Ansatz), ist das der Bernoulli Ansatz? - oder ist der eigentlich nur später für die Wellen?


\begin{align}
\Phi(\rho,\varphi) &= f(\rho) \cdot g(\varphi)
\end{align}

\textbf{Separationsbedingung für 2D-Zylinderkoordinaten:}
Einsetzen von $\Phi(\rho,\varphi) = f(\rho) \cdot g(\varphi)$ in die Laplace-Gleichung und Umformung in die Standardform für das Separationsprinzip:

\begin{align}
\underbrace{\rho^2 \cdot \frac{f''(\rho)}{f(\rho)} + \rho \cdot \frac{f'(\rho)}{f(\rho)}}_{\text{nur von } \rho \text{ abhängig}} + \underbrace{\frac{g''(\varphi)}{g(\varphi)}}_{\text{nur von } \varphi \text{ abhängig}} &= 0
\end{align}

Da beide Terme konstant sein müssen, setzen wir:
\begin{align}
\rho^2 \cdot \frac{f''(\rho)}{f(\rho)} + \rho \cdot \frac{f'(\rho)}{f(\rho)} &= +k_\rho^2\\
\frac{g''(\varphi)}{g(\varphi)} &= -k_\varphi^2
\end{align}

mit der Bedingung $k_\rho^2 + k_\varphi^2 = 0$ bzw. $k_\rho^2 = -k_\varphi^2$.

Dies führt zu den separierten Differentialgleichungen:
\begin{align}
\frac{d^2 f}{d\rho^2} + \frac{1}{\rho}\frac{df}{d\rho} - \frac{k_\rho^2}{\rho^2}f &= 0 \quad \text{(Radiale DGL)}\\
\frac{d^2 g}{d\varphi^2} + k_\varphi^2 g &= 0 \quad \text{(Azimutale DGL)}
\end{align}
mit den Separationskonstanten $k_\rho$ und $k_\varphi$.

\textbf{Lösung der azimutalen Gleichung:}
\begin{align}
\frac{d^2 g}{d\varphi^2} + k_\varphi^2 g &= 0
\end{align}

\textbf{Fall 1:} $k_\varphi^2 > 0$ (d.h. $k_\varphi = n$ reell)
\begin{align}
g(\varphi) &= \begin{Bmatrix} \cos(n\varphi) \\ \sin(n\varphi) \end{Bmatrix}
\end{align}
Periodizitätsbedingung: $g(\varphi + 2\pi) = g(\varphi)$ $\Rightarrow$ $n \in \mathbb{Z}$

\textbf{Fall 2:} $k_\varphi^2 = 0$
\begin{align}
g(\varphi) &= \begin{Bmatrix} 1 \\ \varphi \end{Bmatrix}
\end{align}
Für Periodizität muss der $\varphi$-Term wegfallen $\Rightarrow$ $g(\varphi) = \text{const}$

\textbf{Fall 3:} $k_\varphi^2 < 0$ (d.h. $k_\varphi = i\kappa$ imaginär)
\begin{align}
g(\varphi) &= \begin{Bmatrix} e^{\kappa\varphi} \\ e^{-\kappa\varphi} \end{Bmatrix}
\end{align}
Diese Lösung ist \textbf{nicht periodisch} und wird daher bei Zylinderkoordinaten \textbf{verworfen}.

\textbf{Lösung der radialen Gleichung:}
Mit $k_\rho^2 = -k_\varphi^2$ erhalten wir:

\textbf{Fall 1:} $n \neq 0$ (d.h. $k_\rho^2 = -n^2 < 0$)
\begin{align}
\frac{d^2 f}{d\rho^2} + \frac{1}{\rho}\frac{df}{d\rho} + \frac{n^2}{\rho^2}f &= 0
\end{align}
Lösung: $f(\rho) = \begin{Bmatrix} \rho^n \\ \rho^{-n} \end{Bmatrix}$

\textbf{Fall 2:} $n = 0$ (d.h. $k_\rho^2 = 0$)
\begin{align}
\frac{d^2 f}{d\rho^2} + \frac{1}{\rho}\frac{df}{d\rho} &= 0
\end{align}
Lösung: $f(\rho) = \begin{Bmatrix} 1 \\ \ln\rho \end{Bmatrix}$

\textbf{Allgemeine Lösung:}
\begin{align}
\Phi(\rho,\varphi) &= \begin{Bmatrix} 1 \\ \ln\rho \end{Bmatrix} \cdot \begin{Bmatrix} 1 \\ \varphi \end{Bmatrix} + \begin{Bmatrix} \rho^n \\ \rho^{-n} \end{Bmatrix} \cdot \begin{Bmatrix} \cos(n\varphi) \\ \sin(n\varphi) \end{Bmatrix}
\end{align}



\subsection{Magnetostatik}

\textbf{Hochpermeable Materialien:}\\
Für $\mu \gg \mu_0$ wird die magnetische Feldstärke stark reduziert, da das Material die magnetischen Feldlinien effizient führt.

\begin{align}
\mu \rightarrow \infty \quad \Rightarrow \quad \vec{H} \rightarrow 0 \quad \text{(bei endlichem } \vec{B}\text{)}
\end{align}




\textbf{Biot-Savart:}\\
Magnetfeld $\vec{H}(\vec{r}_p)$ erzeugt durch stromdurchflossenen Leiter.

\begin{align}
\vec{H}(\vec{r}_p) = \frac{I}{4\pi} \int\limits_C \frac{d\vec{l} \times \vec{r}_d}{|\vec{r}_d|^3}
\end{align}


Aufbau des Magnetfeld-Problems:
\begin{itemize}
\item $\vec{r}_p$ -- Ortsvektor zum Aufpunkt P (wo wir das Magnetfeld berechnen wollen)
\item $\vec{r}_q \text{ oder }\vec{r}'$ -- Ortsvektor zum Quellpunkt Q (Punkt auf dem stromdurchflossenen Leiter)
\item $d\vec{l} = d\vec{r}_q$ -- Leiterelement, infinitesimales Vektorelement entlang des Leiters, zeigt in Richtung des Stroms
\item $\vec{r}_d = \vec{r}_p - \vec{r}_q$ -- Abstandsvektor vom Quellpunkt Q zum Aufpunkt P
\item $d\vec{l} \times \vec{r}_d$ -- Kreuzprodukt: Bestimmt die Richtung des Magnetfeldbeitrags (senkrecht zu beiden Vektoren)
\item $|\vec{r}_d|^3$ -- Abstand im Nenner: Sorgt für die $1/r^2$-Abhängigkeit des Magnetfelds (mal ein zusätzlicher Faktor $r$ aus der Vektornormierung)
\item $C$ -- Integrationspfad entlang des stromdurchflossenen Leiters
\end{itemize}


Beispiel Segment: Von $(a,0,0)$ bis $(a,a,0)$:

\begin{align}
d\vec{r}_q(t) &= (a, t, 0) \quad \text{mit } t \in [0,a] \text{ parametrisiert } C \\
d\vec{l} &= \frac{d\vec{r}_q}{dt} \, dt
\end{align}







\textbf{Gegeninduktivität:}\\
Die Gegeninduktivität $M$ beschreibt die magnetische Kopplung zwischen zwei Stromschleifen.

\begin{align}
M = \frac{\Psi}{I}
\end{align}

Dabei ist:
\begin{itemize}
\item Induzierte Spannung: $U_2 = -M \frac{dI_1}{dt}$ (Faraday'sches Induktionsgesetz)
\end{itemize}
