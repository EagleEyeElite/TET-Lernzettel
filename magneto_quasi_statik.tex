\section{Magnetoquasistik MQS}

Vereinfachte Maxwell-Gleichungen ($\frac{\partial \vec{D}}{\partial t} \ll \vec{J}$):

\begin{itemize}
\item $\text{rot } \underline{\vec{E}} + j\omega \underline{\vec{B}} = 0$
\item $\text{rot } \underline{\vec{H}} = \underline{\vec{J}}$
\end{itemize}

Bedingung: $\frac{\omega \varepsilon}{\kappa} \ll 1$\\
$\rightarrow$ Herleitung: $j\omega \underline{\vec{D}} \ll \underline{\vec{J}}$





\textbf{Diffusionsgleichung:} (MQS; $\kappa$, $\mu$ konst.; $\text{div } \vec{E} = \text{div } \vec{H} = 0$) \hfill $\rightarrow$ Herleitung: $\text{rot } \vec{E} + \frac{\partial \vec{B}}{\partial t} = 0$ und
\begin{flushright}
\phantom{$\rightarrow$ Herleitung: }$\text{rot } \vec{H} - \frac{\partial \vec{D}}{\partial t} = \vec{J}$
\end{flushright}


Zeitbereich, z.B. 1D-Funktion $H(x,t)$:
\begin{equation}
\frac{\partial^2 H}{\partial x^2} - \kappa\mu \frac{\partial H}{\partial t} = 0
\end{equation}


Fourier-Transformation (Frequenzbereich):
\begin{align}
\frac{d^2 \underline{H}}{dx^2} - \underline{p}^2 \underline{H} &= 0 \quad \text{mit } \underline{p}^2 = j\omega\kappa\mu \\
\Delta \underline{\vec{H}} - \underline{p}^2 \underline{\vec{H}} &= 0 \\
\Delta \underline{\vec{E}} - \underline{p}^2 \underline{\vec{E}} &= 0
\end{align}


Laplace-Transformation $\underline{H}(x, \underline{s})$ (Anfangsbedingung $H(x, t = 0) = 0$):

\begin{equation}
\frac{\partial^2 \underline{H}(x,\underline{s})}{\partial x^2} - \underline{p}^2 \underline{H}(x,\underline{s}) = 0 \quad \text{mit } \underline{p}^2 = \kappa\mu \underline{s}
\end{equation}

Allgemeine Lösung (für Fourier- und Laplace-Transformation mit Anfangsbedingung $H(x, t = 0) = 0$):

$\rightarrow$ siehe \hyperlink{helmholtz_allgemeine_loesung}{allgemeine Helmholtz-Gleichung}

\textbf{Feldabfall (bei MQS):}


Exponentieller Abfall im quasistatischen Fall:
\begin{align}
|\underline{H}(d)| = H_0 e^{-\frac{d}{\delta}}
\end{align}


Feldverhältnis der Ränder: 
\begin{align}
\frac{H(d)}{H(0)}
\end{align}


\textbf{Poyntingscher Satz:}
\begin{align}
-\frac{\partial w_{em}}{\partial t} = \text{div}(\vec{S}) + p_V
\end{align}
