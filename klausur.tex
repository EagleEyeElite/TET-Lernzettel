\section{Klausur}

\begin{itemize}
\item Klausurdauer: 2,5 h
\item Keine Hilfsmittel: Nur Stift, kein Taschenrechner, kein Hilfszettel, keine Formelsammlung etc.
\item 2. Termin: 07.10.25, 8:30-11:00 HE101
\item Format: 1) kurze unabhängige Aufgaben (wie HA5)\\
2) 3) 4) Rechenaufgaben (drei von den 5 Themen zu rechnen - halbe Stunde pro Aufgabe)
\item Besselfunktion kommt nicht ran (nicht lernen)
\item Wellen: wahrscheinlich keine komplette Rechenaufgabe, aber ein, zwei Teilaufgaben in 1) - z.B. Polarisation/Dipol, gehört haben, aber nicht zum Rechnen $\rightarrow$ Wenn in der Aufgabe: ohne Rechnung $\rightarrow$ Dann auch wirklich nur Ergebnis.
\item Meiste kommt aus der Übung. Wenn die Übung alle verstanden haben, dann gut für Klausur.
\item Altklausuren, die 10 Jahre alt sind, sind noch einigermaßen okay aktuell (reicht zum Lernen).
\item konstaten auswendig lernen nicht notwendig (außer höchstens C). Z.B.: Lösung 4 $\epsilon_0$ reicht aus
\item Einheitscheck immer ganz gut
\item Maxwell wird immer abgefragt: wie 1. Aufgabe in der Problekalusur, oder auch in Frequenz, oder Integralform
\item Herleitung für Bedingung für MQS könnte abgefragt werden: $\frac{\omega \varepsilon}{\kappa} \ll 1$
\item rot H = dD/dt + J -> (d/dt zu jw)  - Trafo der Maxwell-Gleichung mit Fourier könnte auch abgefragt werden
\item Biot-Savart müssen wir nicht auswendig kennen für die Klausur, auch nicht herleiten können, aber anwenden können. Integrallösung üben muss nicht geübt werden. Nur aufschreiben, verstehen und anwenden.
\item div A = 0 -- Coulomb-Eichung


\end{itemize}