\documentclass{article}
\usepackage{amsmath}
\usepackage{amssymb}
\usepackage{graphicx}
\usepackage{geometry}
\usepackage{placeins}
\usepackage{tikz}
\usepackage{svg}
\usepackage{hyperref}
\usepackage{mathtools}
\usepackage{float}
\usetikzlibrary{arrows,positioning,shapes.geometric}

\geometry{
  a4paper,
  margin=1.5cm,
  top=1cm,
  includefoot,
  includehead
}
\usepackage{titling}
\setlength{\droptitle}{-2cm}

\usepackage{parskip}
\setlength{\parindent}{0pt}
\title{TET Lernzettel}
\author{Conrad Klaus \\ 
\href{https://github.com/EagleEyeElite/TET-Lernzettel}{github.com/EagleEyeElite/TET-Lernzettel} \\
\href{https://www.overleaf.com/read/nczbsvtfhhtx\#5f73ee}{overleaf.com/read/nczbsvtfhhtx\#5f73ee}}
\date{}

\definecolor{myblue}{RGB}{0,102,204}
\definecolor{myred}{RGB}{204,0,0}


\begin{document}
\maketitle

\textit{Hinweis: Ich garantiere nicht für Vollständigkeit oder Richtigkeit, mir persönlich hat die Übersicht aber sehr geholfen. Die Formeln sollten alle richtig sein.}

\vspace{0.5cm}

\section{Symbole}

\textbf{Dach-Symbol ($\hat{\ }$):}
\begin{itemize}
\item Mit Dach: $\hat{\Phi}$ -- Konstante
\item Ohne Dach: $\Phi(x)$ -- Funktion
\end{itemize}


\textbf{Komplexe Größen:}
\begin{itemize}
\item $\underline{Z}$ -- Komplexe Größe
\item $Z' = \text{Re}(\underline{Z})$ -- Realteil (ein Strich)
\item $Z'' = \text{Im}(\underline{Z})$ -- Imaginärteil (zwei Striche)
\item $\underline{Z}^* = Z' - jZ''$ -- Komplex konjugiert
\end{itemize}



% alter code, wahrscheinlich falsch, aber irgendwas war mit der fourie trafo
%\textbf{Phasor:}
%\begin{itemize}
%\item $v(t) = V_0 \cdot \cos(\omega t + \varphi) \overset{\mathcal{F}}{\circ\text{---}\bullet} \underline{V} = V_0 \cdot e^{j\varphi}$ -- Phasor, für feste Kreisfrequenz $\omega$
%\item $Z(t) \overset{\mathcal{F}}{\circ\text{---}\bullet} \underline{Z}$ -- Phasor (Fourier-transformierte komplexe Amplitude)
%\item $\underline{Z} = Z_0 e^{j\varphi}$ -- Exponentialdarstellung des Phasors
%\item $Z(t) = \text{Re}\{\underline{Z} e^{j\omega t}\}$ -- Rücktransformation (physikalische Größe)
%\end{itemize}
%Der Phasor beschreibt nur Amplitude und Phase bei einer festen Frequenz.
\textbf{Phasor:}
\begin{itemize}
\item $Z = Z(t) = Z_0 \cdot \cos(\omega t + \varphi) \leftrightarrow \underline{Z} = Z_0 e^{j\varphi}$ -- Phasor, für feste Kreisfrequenz $\omega$
\item $Z(t) = \text{Re}\{\underline{Z} e^{j\omega t}\} = \text{Re}\{Z_0 e^{j\varphi + j\omega t}\}$ -- Rücktransformation (physikalische Größe)
\end{itemize}
Der Phasor beschreibt nur Amplitude und Phase bei einer festen Frequenz.


\textbf{Balken-Symbol ($\overline{\phantom{x}}$):}
\begin{itemize}
\item $\overline{X}$ -- zeitlicher Mittelwert von X
\item $\overline{|\vec{J}|^2} = \overline{|\vec{J}(t)|^2} = \overline{J_0^2 \cos^2(\omega t)} = \frac{1}{2}J_0^2 = \frac{1}{2}\underline{\vec{J}} \cdot \underline{\vec{J}}^*$ -- zeitlicher Mittelwert des Quadrats der Stromdichte
\item $I_{\text{eff}} = \sqrt{\overline{I^2(t)}} = \sqrt{\overline{I_0^2 \cos^2(\omega t)}} = \sqrt{\frac{1}{2}I_0^2} = \frac{1}{\sqrt{2}}I_0$ -- Effektivstrom (RMS-Wert)
\item $\overline{\vec{S}} = \overline{\vec{E} \times \vec{H}} = \frac{1}{2}\text{Re}\{\underline{\vec{E}} \times \underline{\vec{H}}^*\}$ -- zeitlicher Mittelwert des Poynting-Vektors
\item $\overline{p_V} = \frac{1}{2}\kappa_{\text{äq}}|\underline{\vec{E}}|^2$ -- zeitgemittelte Volumenverlustleistungsdichte
\end{itemize}


\textbf{Hierarchie verschiedene Bezugsgrößen:}
\begin{itemize}
\item $P_V = \int\limits_V p_V \, dV$ -- Gesamte Verlustleistung \hfill $[P_V] = \text{W}$
\item $P_V' = \int\limits_A p_V \, d\vec{A}$ -- Verlustleistung pro Längeneinheit \hfill $[P_V'] = \frac{\text{W}}{\text{m}}$
\item $P_V'' = \int\limits_s p_V \, d\vec{s}$ -- Verlustleistung pro Flächeneinheit \hfill $[P_V''] = \frac{\text{W}}{\text{m}^2}$
\item $p_V = \vec{J} \cdot \vec{E} = \frac{|\vec{J}|^2}{\kappa}$ -- Volumenverlustleistungsdichte \hfill $[p_V] = \frac{\text{W}}{\text{m}^3}$
\end{itemize}


\textbf{Maxwell-Gleichungen:}
\begin{align}
\text{div } \vec{D} &= \varrho \quad \text{(Gaußsches Gesetz)} \\
\text{div } \vec{B} &= 0 \quad \text{(Quellenfreiheit des Magnetfeldes)} \\
\text{rot } \vec{E} + \frac{\partial \vec{B}}{\partial t} &= 0 \quad \text{(Faradaysches Induktionsgesetz)} \\
\text{rot } \vec{H} - \frac{\partial \vec{D}}{\partial t} &= \vec{J} \quad \text{(Ampèresches Gesetz/ Durchflutungsgesetz)}
\end{align}

Fourier-Transformation Beispiele:
\begin{align}
\text{rot } \vec{E} + \frac{\partial \vec{B}}{\partial t} = 0 &\overset{\mathcal{F}}{\circ\text{---}\bullet} \text{rot } \underline{\vec{E}} + j\omega \underline{\vec{B}} = 0 \\
\text{rot } \vec{H} - \frac{\partial \vec{D}}{\partial t} = \vec{J} &\overset{\mathcal{F}}{\circ\text{---}\bullet} \text{rot } \underline{\vec{H}} - j\omega \underline{\vec{D}} = \underline{\vec{J}}
\end{align}

Integralform Beispiele:
\begin{align}
\oint\limits_{\partial V} \vec{D} \cdot d\vec{A} &= \int\limits_V \varrho \, dV \\
\oint\limits_{\partial A} \vec{H} \cdot d\vec{s} - \int\limits_A \frac{\partial \vec{D}}{\partial t} \cdot d\vec{A} &= \int\limits_A \vec{J} \cdot d\vec{A}
\end{align}


\textbf{Andere Gleichungen:}

\begin{itemize}
\item Poisson-Gleichung \hfill $\rightarrow$ Herleitung: $\text{div } \vec{D} = \varrho$
\begin{align}
\Delta \Phi &= -\frac{\varrho}{\varepsilon}
\end{align}

\item Laplace-Gleichung ($\varrho = 0$)
\begin{align}
\Delta \Phi &= 0
\end{align}

\item Allgemeine Helmholtz-Gleichung (beliebige Felder $\underline{\vec{A}}$)
\begin{align}
\Delta \underline{\vec{A}} + \underline{k}^2 \underline{\vec{A}} &= 0 \quad \text{mit } \underline{k} = \beta - j\alpha
\end{align}
$\rightarrow \beta$ Phasenkonstante, $\alpha$ Dämpfungskonstante


\hypertarget{helmholtz_allgemeine_loesung}{Allgemeine Lösung (1D-Fall):}\\
\begin{equation}
\underline{A}(x) = \underline{C}_1 e^{\underline{k} \cdot x} + \underline{C}_2 e^{-\underline{k} \cdot x}
\end{equation}
\begin{equation}
\underline{A}(x) = \underline{K}_1 \cosh(\underline{k} \cdot x) + \underline{K}_2 \sinh(\underline{k} \cdot x)
\end{equation}
\textit{Hinweis:} Bei Helmholtz-Gleichung immer 1D-Abhängigkeit (Bsp.: d/dy = d/dz = 0), sonst müsste man rot() mitbeachten.






\item Inhomogen oder Quellterm ??? in dieser Sektion gibt es noch Probleme :((: 
\begin{equation}
\Delta \underline{\vec{A}} + k^2(\vec{r}) \underline{\vec{A}} = \vec{Q}
\end{equation}
Inhomogen ($\varepsilon(\vec{r})$), Quellterm ($\vec{J}_e \neq 0$): 
\begin{equation}
\Delta\vec{H} + k^2(\vec{r})\vec{H} = -\text{rot } \vec{J} \quad \text{mit } k(\vec{r}) = \omega\sqrt{\mu_0\varepsilon(\vec{r})}
\end{equation}


\end{itemize}


\textbf{Elektrodynamik:}
\begin{itemize}
\item $\Phi = k_c \frac{Q}{|\vec{r}|}$ -- Elektrisches Potential \hfill $[\Phi] = \text{V}$
\item $\vec{E} = -\text{grad } \Phi$ -- Elektrisches Feld (elektrostatischer Fall) \hfill $[\vec{E}] = \frac{\text{N}}{\text{C}} = \frac{\text{V}}{\text{m}}$
\item $\vec{D} = \varepsilon \vec{E}$ -- Elektrische Flussdichte (Verschiebungsdichte/ Verschiebungsfeld) \hfill $[\vec{D}] = \frac{\text{C}}{\text{m}^2} = \frac{\text{As}}{\text{m}^2}$
\item $\vec{J} = \kappa \vec{E}$ -- elektrische Stromdichte (Volumenstromdichte) \hfill $[\vec{J}] = \frac{\text{A}}{\text{m}^2}$
\item $\vec{K} = \vec{J} \cdot d$ -- Oberflächenstromdichte \hfill $[\vec{K}] = \frac{\text{A}}{\text{m}}$
\item $\vec{H}$ -- Magnetisches Feld \hfill $[\vec{H}] = \frac{\text{A}}{\text{m}}$
\item $\vec{A}$ -- Vektorpotential (magnetisches Wirbelpotential) \hfill $[\vec{A}] = \frac{\text{Wb}}{\text{m}} = \frac{\text{Vs}}{\text{m}}$
\item $\vec{B} = \mu \vec{H} = \text{ rot }\vec{A}$ -- Magnetische Flussdichte \hfill $[\vec{B}] = \text{T} = \frac{\text{Wb}}{\text{m}^2} = \frac{\text{Vs}}{\text{m}^2}$
\item $\psi \text{ oder } \Psi \text{ oder } \Phi = \int \vec{B} \cdot d\vec{A}$ -- Magnetischer Fluss \hfill $[\psi] = \text{Wb} = \text{Vs}$
\item $M = \frac{\psi}{I}$ -- Gegeninduktivität \hfill $[M] = \text{H} = \frac{\text{Wb}}{\text{A}} = \frac{\text{Vs}}{\text{A}}$
\item $\varrho$ oder $\rho$ -- Raumladungsdichte \hfill $[\varrho] = \frac{\text{C}}{\text{m}^3} = \frac{\text{As}}{\text{m}^3}$
\item $\sigma$ -- Flächenladungsdichte \hfill $[\sigma] = \frac{\text{C}}{\text{m}^2} = \frac{\text{As}}{\text{m}^2}$
\item $\lambda$ -- Linienladungsdichte \hfill $[\lambda] = \frac{\text{C}}{\text{m}} = \frac{\text{As}}{\text{m}}$
\item $\omega = 2\pi f$ -- Kreisfrequenz \hfill $[\omega] = \frac{1}{\text{s}}$
\item $k = \omega \sqrt{\epsilon\mu}$ -- Wellenzahl (verlustfreie Medien)  \hfill $[k] = \frac{1}{\text{m}}$
\item $\underline{p} = \sqrt{j\omega\kappa\mu} = \sqrt{\frac{\omega\kappa\mu}{2}}(1+j) = \frac{1+j}{\delta}$ -- Propagationskonstante (MQS, verlustbehaftete Medien) \hfill  $[\underline{p}] = \frac{1}{\text{m}}$
\item $\delta = \sqrt{\frac{2}{\omega\kappa\mu}}$ -- Eindringtiefe/ Skintiefe (Tiefe bei $e^{-1} \approx 37\%$ Amplitude) \hfill $[\delta] = \text{m}$
\item $p_V = \vec{J} \cdot \vec{E} = \frac{|\vec{J}|^2}{\kappa}$ -- Volumenverlustleistungsdichte \hfill $[p_V] = \frac{\text{W}}{\text{m}^3} = \frac{\text{A} \text{V}}{\text{m}^3} $
\item $P_V = \int_V p_V \, dV$ -- Gesamte Verlustleistung (V für Verlust)\hfill $[P_V] = \text{W} = \text{A}\text{V}$
\item $\vec{S} = \vec{E} \times \vec{H}$ -- Poynting-Vektor (Energieflussdichte) \hfill $[\vec{S}] = \frac{\text{W}}{\text{m}^2}= \frac{\text{A} \text{V}}{\text{m}^2} $
\item $Z = \sqrt{\frac{\mu}{\varepsilon}}$ -- Wellenimpedanz \hfill $[Z] = \Omega = \frac{\text{V}}{\text{A}}$
\end{itemize}


\textbf{Materialparameter:}
\begin{itemize}
\item $\varepsilon = \varepsilon_0 \varepsilon_r$ -- Permittivität (Dielektrizitätskonstante) \hfill $[\varepsilon] = \frac{\text{F}}{\text{m}} = \frac{\text{As}}{\text{Vm}}$
\item $\mu = \mu_0 \mu_r$ -- Permeabilität (Magnetische Durchlässigkeit) \hfill $[\mu] = \frac{\text{H}}{\text{m}} = \frac{\text{Vs}}{\text{Am}}$
\item $\kappa$ oder $\sigma$ -- Elektrische Leitfähigkeit \hfill $[\kappa] = \frac{\text{S}}{\text{m}} = \frac{\text{A}}{\text{Vm}}$
\item $\kappa_{\text{äq}} = \omega\varepsilon''\tan(\delta)$ -- äquivalente Leitfähigkeit \hfill $[\kappa_{\text{äq}}] = \frac{\text{S}}{\text{m}} = \frac{\text{A}}{\text{Vm}}$
\end{itemize}


\textbf{Elektromagnetische Energien:}
\begin{itemize}
\item $w_e = \frac{1}{2} \vec{E} \cdot \vec{D}$ -- Elektrische Energiedichte \hfill $[w_e] = \frac{\text{J}}{\text{m}^3} = \frac{\text{Ws}}{\text{m}^3}$
\item $\displaystyle W_e = \int\limits_V w_e \, dV$ \quad -- Elektrische Energie \hfill $[W_e] = \text{J} = \text{Ws}$
\begin{flalign*}
&= \frac{1}{2} \left(\int\limits_V \varrho \Phi \, dV - \oint\limits_{\partial V} \Phi \vec{D} \cdot d\vec{A}\right) \xrightarrow[\partial V \to \infty]{\text{lok. } \varrho} \frac{1}{2} \int\limits_V \varrho \Phi \, dV   \quad \text{ (bei Elektrostatik)} &&
\end{flalign*}
\item $w_m = \frac{1}{2} \vec{H} \cdot \vec{B}$ -- Magnetische Energiedichte \hfill $[w_m] = \frac{\text{J}}{\text{m}^3} = \frac{\text{Ws}}{\text{m}^3}$
\item $W_m = \int\limits_V w_m \, dV$ -- Magnetische Energie \hfill $[W_m] = \text{J} = \text{Ws}$
\item $w_{em} = w_e + w_m$ -- Gesamte Energiedichte \hfill $[w_{em}] = \frac{\text{J}}{\text{m}^3} = \frac{\text{Ws}}{\text{m}^3}$
\item $W_{em} = W_e + W_m$ -- Gesamte Energie \hfill $[W_{em}] = \text{J} = \text{Ws}$
\end{itemize}


\textbf{Stichwörter:}
\begin{itemize}
\item Lineare, Isotrope und Homogene Materie
\item Isotrop: Materialkonstanten richtungsunabhängig (über Materialgleichungen verknüpfte Größen haben dieselbe Richtung)
\item homogenes, verlustfreies Medium $\Leftrightarrow$ $\epsilon$, $\mu$ konst.; $\kappa = 0$
\item Transversal: quer zur Ausbreitungsrichtung
\end{itemize}


\textbf{Kräfte:}
\begin{itemize}
\item $\vec{F} = Q(\vec{E} + \vec{v} \times \vec{B})$ -- Lorentz-Kraft \hfill $[\vec{F}] = \text{N} = \frac{\text{kg} \cdot \text{m}}{\text{s}^2}$
\item $\vec{F_C} = Q\vec{E}$ -- Coulomb-Kraft ($|\vec{v}| = 0$)
\item $\vec{F_L} = Q(\vec{v} \times \vec{B})$ -- Magnetische Lorentz-Kraft
\end{itemize}


\textbf{Konstanten:}
\begin{itemize}
\item $c = \sqrt{\frac{1}{\mu \varepsilon}}$; $c_0 = \sqrt{\frac{1}{\mu_0 \varepsilon_0}} \approx 3 \cdot 10^8 \frac{\text{m}}{\text{s}}$ -- Lichtgeschwindigkeit \hfill $[c] = \frac{\text{m}}{\text{s}}$
\item $k_c \text{ oder } k = \frac{1}{4\pi\varepsilon_0}$ -- Coulomb-Konstante (für Vakuum) \hfill $[k] = \frac{\text{N} \cdot \text{m}^2}{\text{C}^2} = \frac{\text{Vm}}{\text{As}}$
\item $k_{medium} = \frac{1}{4\pi\varepsilon_0\varepsilon_r}$ -- Coulomb-Konstante (für Medium) \hfill $[k] = \frac{\text{N} \cdot \text{m}^2}{\text{C}^2} = \frac{\text{Vm}}{\text{As}}$
\end{itemize}


\textbf{Mathematische Identitäten:}
\begin{itemize}
\item $\text{div rot } \vec{A} = 0$
\item $\text{rot grad } \Phi = 0$ -- Merksatz: Rot kraut ist tief rot, beide sind null
\item $\text{rot rot } \vec{A} = \text{grad div } \vec{A} - \nabla^2 \vec{A} = \text{grad div } \vec{A} - \Delta \vec{A}$
\item $\text{rot}(\vec{A} - \vec{B}) = \text{rot } \vec{A} - \text{rot } \vec{B}$
\item $\text{rot}(A(\vec{r}) \cdot \vec{B}) = (\text{grad}\,A(\vec{r})) \times \vec{B} + A(\vec{r}) \cdot \text{rot}\,\vec{B}$
\item $\nabla \times (A(\vec{r}) \vec{B}) = (\nabla A(\vec{r})) \times \vec{B} + A(\vec{r}) (\nabla \times \vec{B})$
\item $\nabla \cdot (A(\vec{r}) \vec{B}) = (\nabla A(\vec{r})) \cdot \vec{B}   + A(\vec{r}) (\nabla \cdot \vec{B})$
\item $\int\limits_A \text{rot } \vec{v} \cdot d\vec{A} = \oint\limits_{\partial A} \vec{v} \cdot d\vec{s}$ -- Stokes'scher Integralsatz
\item $\int\limits_V \text{div } \vec{v} \, dV = \oint\limits_{\partial V} \vec{v} \cdot d\vec{A}$ -- Gaußscher Integralsatz
\item $j = \frac{(1+j)^2}{2}$ oder $\sqrt{j} = \frac{1+j}{\sqrt{2}}$
\item $e^{j\theta} = \cos(\theta) + j \sin(\theta)$ -- Eulersche Formel
\item $x^2 + y^2 = r^2$ -- Kreisgleichung
\item $\cos^2(x) + \sin^2(x) = 1$ -- Trigonometrische Identität
\item $x^2 - y^2 = r^2$ -- HyperbelGleichung
\item $\cosh^2(x) - \sinh^2(x) = 1$ -- Hyperbolische Identität
\item $\cosh(\underline{z}) = \cos(j\underline{z})$
\item $\sinh(\underline{z}) = -j\sin(j\underline{z})$
\item $e^x - e^{-x} = 2\sinh(x)$
\item $e^x + e^{-x} = 2\cosh(x)$
\item $(1 + p)^q \approx 1 + pq \quad \text{für } p \ll 1$




\item $\text{div } \vec{A} = \frac{1}{\varrho} \frac{\partial(\varrho A_\varrho)}{\partial \varrho} + \frac{1}{\varrho} \frac{\partial A_\varphi}{\partial \varphi} + \frac{\partial A_z}{\partial z}$ -- Zylinder koo.
\item $\text{grad } \Phi = \frac{\partial \Phi}{\partial \varrho} \vec{e}_\varrho + \frac{1}{\varrho} \frac{\partial \Phi}{\partial \varphi} \vec{e}_\varphi + \frac{\partial \Phi}{\partial z} \vec{e}_z$ -- Zylinder koo.

\item $\int \frac{1}{1+ax} dx = \frac{1}{a} \int \frac{1}{u} du = \frac{1}{a} \ln|u| + C = \frac{1}{a} \ln|1+ax| + C$ \quad\quad (Substitution: $u = 1 + ax \Rightarrow du = a \, dx \Rightarrow dx = \frac{1}{a} du$)
\item $\int \frac{1}{x} dx = \ln|x| + C \quad \Leftrightarrow \quad \frac{d}{dx}[\ln|x|] = \frac{1}{x}$
\end{itemize}




\textbf{Koordinaten:}
\begin{itemize}
\item $\rho$ oder $\varrho$ -- Radiale Koordinate (Abstand von z-Achse), $\rho \geq 0$
\item $\varphi$ oder $\phi$ -- Azimutwinkel in x-y-Ebene, $0 \leq \varphi < 2\pi$
\item $\vartheta$ oder $\theta$ -- Polarwinkel oder Zenitwinkel, $0 \leq \vartheta \leq \pi$
\end{itemize}

Kugelvolumen:
\begin{itemize}
\item $V = \int\limits_V dV = \int\limits_0^R \int\limits_0^{2\pi} \int\limits_0^{\pi} r^2 \sin\vartheta \, d\vartheta \, d\varphi \, dr = \left[\frac{r^3}{3}\right]_0^R \cdot 2\pi \cdot \left[-\cos\vartheta\right]_0^{\pi} = \frac{R^3}{3} \cdot 2\pi  \cdot 2 = \frac{4\pi R^3}{3}$

\item $A = \oint\limits_{\partial V} d\vec{A} = \int\limits_0^{2\pi} \int\limits_0^{\pi} r^2 \sin\vartheta \, d\vartheta \, d\varphi = 4\pi r^2$
\end{itemize}


\textbf{Nabla-Operator:}
\begin{itemize}
\item $\nabla = \left(\frac{\partial}{\partial x}, \frac{\partial}{\partial y}, \frac{\partial}{\partial z}\right)$
\item $\nabla \Phi = \text{grad } \Phi$ -- Gradient (Nabla auf Skalarfeld)
\item $\text{div grad } \Phi = \nabla \cdot (\nabla \Phi) =  \nabla^2 \Phi = \Delta \Phi$ -- Laplace-Operator (Dreieck auf dem Kopf)
\end{itemize}


\textbf{Rot-Operator:}

\begin{itemize}
\item $\vec{a} \times \vec{b} = \begin{pmatrix}
a_2 b_3 - b_2 a_3 \\
a_3 b_1 - b_3 a_1 \\
a_1 b_2 - b_1 a_2
\end{pmatrix}$
\item $\text{rot }\vec{E} =\nabla \times \vec{E} = \begin{pmatrix}
\frac{\partial E_z}{\partial y} - \frac{\partial E_y}{\partial z} \\[0.5em]
\frac{\partial E_x}{\partial z} - \frac{\partial E_z}{\partial x} \\[0.5em]
\frac{\partial E_y}{\partial x} - \frac{\partial E_x}{\partial y}
\end{pmatrix}$
\end{itemize}

\textbf{Matrixschreibweise für Basisfunktionen:}\\
Die geschweifte Klammer-Notation $\begin{Bmatrix} f_1 \\ f_2 \end{Bmatrix}$ steht für eine \textbf{Linearkombination} der Basisfunktionen:
\begin{align}
\begin{Bmatrix} f_1 \\ f_2 \end{Bmatrix} &\equiv A \cdot f_1 + B \cdot f_2
\end{align}

Beispiele:
\begin{align}
\begin{Bmatrix} 1 \\ \ln\rho \end{Bmatrix} \begin{Bmatrix} 1 \\ \varphi \end{Bmatrix} + \begin{Bmatrix} \rho^n \\ \rho^{-n} \end{Bmatrix} \begin{Bmatrix} \cos(n\varphi) \\ \sin(n\varphi) \end{Bmatrix} &= (A_0 + B_0 \ln\rho)(C_0 + D_0 \varphi) \\
&\quad + (A_1 \rho^n + B_1 \rho^{-n})(C_1 \cos(n\varphi) + D_1 \sin(n\varphi))
\end{align}


\section{Statik}

Vereinfachte Maxwell-Gleichungen ($\frac{\partial}{\partial t} = 0$):
\begin{itemize}
\item $\text{div } \vec{D} = \rho$
\item $\text{rot } \vec{H} = \vec{J}$
\end{itemize}

Hier aufpassen: keine poission Gleichung lösen (da keine Ladung im Inneren des Raums ist, selbst Flächenladung braucht keine Poisson-Gleichung)

Randwertproblem/ Integral kommt auf jeden Fall als Rechenaufgabe in Klausur -> Also Potenzialproblem auf jeden Fall!

\subsection{Stetigkeitsbedingungen}

\begin{figure}[H]
\centering
\includesvg[width=0.45\textwidth]{1.svg}
\hfill
\includesvg[width=0.45\textwidth]{2.svg}
\end{figure}

\begin{itemize}
\item $D_{n2} - D_{n1} = \sigma$
\item $B_{n2} - B_{n1} = 0$
\item $E_{t2} - E_{t1} = 0$
\item $H_{t2} - H_{t1} = \vec{K}$
\end{itemize}

\subsection{Elektrostatik}

\subsubsection{Laplace-Operator Bedeutung}
\textbf{Physikalisch:}
\begin{align}
\Delta \Phi &= \nabla \cdot (\nabla \Phi) = \nabla \cdot \vec{E}
\end{align}
$\Delta \Phi = 0$ bedeutet: Keine Quellen/Senken, harmonischer Verlauf, ladungsfreies Gebiet


\textbf{Elektrostatik:}
\begin{align}
\text{Ladungsfrei:} \quad &\Delta \Phi = 0 \quad \text{(Laplace)}\\
\text{Mit Ladungen:} \quad &\Delta \Phi = -\frac{\rho}{\varepsilon} \quad \text{(Poisson)}\\
\text{Stationäre Strömung:} \quad &\text{div } \vec{J} = 0 \quad \text{(Potentialgleichung)}
\end{align}


\subsubsection{Laplace-Gleichung}
Allgemeine Form:
\begin{align}
\Delta\Phi(x, y, z) = \text{div grad } \Phi(x, y, z) = 0 \qquad (\text{div } \vec{D} = \varrho \text{, mit }\varrho = 0)
\end{align}




Kartesische Koordinaten $(x, y, z)$:
\begin{align}
\Delta \Phi &= \frac{\partial^2 \Phi}{\partial x^2} + \frac{\partial^2 \Phi}{\partial y^2} + \frac{\partial^2 \Phi}{\partial z^2} = 0
\end{align}
Zylinderkoordinaten $(\rho, \varphi, z)$:
\begin{align}
\Delta \Phi &= \frac{\partial^2 \Phi}{\partial \rho^2} + \frac{1}{\rho}\frac{\partial \Phi}{\partial \rho} + \frac{1}{\rho^2}\frac{\partial^2 \Phi}{\partial \varphi^2} + \frac{\partial^2 \Phi}{\partial z^2} = 0
\end{align}












\subsubsection{Separationsansatz}
Grundidee: Die Lösung einer partiellen Differentialgleichung wird als Produkt von Funktionen einzelner Variablen angesetzt:

\begin{align}
\Phi(x, y, z) &= f(x) \cdot g(y) \cdot h(z)\\
\Delta\Phi(x, y, z) &= \frac{d^2 f(x)}{dx^2} g(y)h(z) + f(x) \frac{d^2 g(y)}{dy^2} h(z) + f(x)g(y) \frac{d^2h(z)}{dz^2} = 0\\
% Division durch Φ(x,y,z)
&= \underbrace{\frac{1}{f} \frac{d^2 f}{dx^2}}_{K_x} +
\underbrace{\frac{1}{g} \frac{d^2 g}{dy^2}}_{K_y} +
\underbrace{\frac{1}{h} \frac{d^2h}{dz^2}}_{K_z} = 0
\end{align}

% Einzellösungen
Einzellösungen:
% Fall 1: Kx = 0
\begin{align}
f(x) &= \begin{Bmatrix} 1 \\ x \end{Bmatrix} \quad \text{mit } K_x = 0 \tag{3.43}\\
% Fall 2: Kx > 0
f(x) &= \begin{Bmatrix} e^{k_x x} \\ e^{-k_x x} \end{Bmatrix} \quad \text{mit } K_x = k_x^2 > 0 \tag{3.44}\\
% Fall 3: Kx < 0
f(x) &= \begin{Bmatrix} \sin(k_x x) \\ \cos(k_x x) \end{Bmatrix} \quad \text{mit } K_x = -k_x^2 < 0 \tag{3.45}
\end{align}


% Erklärung der Notation
Notation:
\begin{itemize}
    \item Die Konstanten $K$: Separationskonstante, als Quotient $f''/f$ definiert. 
    \item Die Konstanten $k$: Skalierungsfaktor (ähnlich wie später die Wellenzahl $k$). Man beachte, dass je nach Funktion $K_x = \pm k_x^2$ gilt.
    \item Die Separationsgleichung: $K_x + K_y + K_z = 0$.
\end{itemize}


\textbf{Flächenladung:}


Integration des Gaußschen Gesetzes:

\begin{align}
\int_V \nabla \cdot \vec{D} \, dV = \oint_{\partial V} \vec{D} \cdot \hat{n} \, dA = \int_V \rho \, dV = Q_{\text{eingeschlossen}}
\end{align}

Mantelbeiträge (Höhe $h \to 0$):
\begin{align}
D_y(y=0^+) \cdot dA - D_y(y=0^-) \cdot dA = \sigma(x,z) \cdot dA
\end{align}
\begin{align}
\Leftrightarrow D_y(y=0^+) - D_y(y=0^-) = \sigma(x,z)
\end{align}






















\subsubsection{Separationsansatz}

? (Bernoulli-Ansatz), ist das der Bernoulli Ansatz? - oder ist der eigentlich nur später für die Wellen?


\begin{align}
\Phi(\rho,\varphi) &= f(\rho) \cdot g(\varphi)
\end{align}

\textbf{Separationsbedingung für 2D-Zylinderkoordinaten:}
Einsetzen von $\Phi(\rho,\varphi) = f(\rho) \cdot g(\varphi)$ in die Laplace-Gleichung und Umformung in die Standardform für das Separationsprinzip:

\begin{align}
\underbrace{\rho^2 \cdot \frac{f''(\rho)}{f(\rho)} + \rho \cdot \frac{f'(\rho)}{f(\rho)}}_{\text{nur von } \rho \text{ abhängig}} + \underbrace{\frac{g''(\varphi)}{g(\varphi)}}_{\text{nur von } \varphi \text{ abhängig}} &= 0
\end{align}

Da beide Terme konstant sein müssen, setzen wir:
\begin{align}
\rho^2 \cdot \frac{f''(\rho)}{f(\rho)} + \rho \cdot \frac{f'(\rho)}{f(\rho)} &= +k_\rho^2\\
\frac{g''(\varphi)}{g(\varphi)} &= -k_\varphi^2
\end{align}

mit der Bedingung $k_\rho^2 + k_\varphi^2 = 0$ bzw. $k_\rho^2 = -k_\varphi^2$.

Dies führt zu den separierten Differentialgleichungen:
\begin{align}
\frac{d^2 f}{d\rho^2} + \frac{1}{\rho}\frac{df}{d\rho} - \frac{k_\rho^2}{\rho^2}f &= 0 \quad \text{(Radiale DGL)}\\
\frac{d^2 g}{d\varphi^2} + k_\varphi^2 g &= 0 \quad \text{(Azimutale DGL)}
\end{align}
mit den Separationskonstanten $k_\rho$ und $k_\varphi$.

\textbf{Lösung der azimutalen Gleichung:}
\begin{align}
\frac{d^2 g}{d\varphi^2} + k_\varphi^2 g &= 0
\end{align}

\textbf{Fall 1:} $k_\varphi^2 > 0$ (d.h. $k_\varphi = n$ reell)
\begin{align}
g(\varphi) &= \begin{Bmatrix} \cos(n\varphi) \\ \sin(n\varphi) \end{Bmatrix}
\end{align}
Periodizitätsbedingung: $g(\varphi + 2\pi) = g(\varphi)$ $\Rightarrow$ $n \in \mathbb{Z}$

\textbf{Fall 2:} $k_\varphi^2 = 0$
\begin{align}
g(\varphi) &= \begin{Bmatrix} 1 \\ \varphi \end{Bmatrix}
\end{align}
Für Periodizität muss der $\varphi$-Term wegfallen $\Rightarrow$ $g(\varphi) = \text{const}$

\textbf{Fall 3:} $k_\varphi^2 < 0$ (d.h. $k_\varphi = i\kappa$ imaginär)
\begin{align}
g(\varphi) &= \begin{Bmatrix} e^{\kappa\varphi} \\ e^{-\kappa\varphi} \end{Bmatrix}
\end{align}
Diese Lösung ist \textbf{nicht periodisch} und wird daher bei Zylinderkoordinaten \textbf{verworfen}.

\textbf{Lösung der radialen Gleichung:}
Mit $k_\rho^2 = -k_\varphi^2$ erhalten wir:

\textbf{Fall 1:} $n \neq 0$ (d.h. $k_\rho^2 = -n^2 < 0$)
\begin{align}
\frac{d^2 f}{d\rho^2} + \frac{1}{\rho}\frac{df}{d\rho} + \frac{n^2}{\rho^2}f &= 0
\end{align}
Lösung: $f(\rho) = \begin{Bmatrix} \rho^n \\ \rho^{-n} \end{Bmatrix}$

\textbf{Fall 2:} $n = 0$ (d.h. $k_\rho^2 = 0$)
\begin{align}
\frac{d^2 f}{d\rho^2} + \frac{1}{\rho}\frac{df}{d\rho} &= 0
\end{align}
Lösung: $f(\rho) = \begin{Bmatrix} 1 \\ \ln\rho \end{Bmatrix}$

\textbf{Allgemeine Lösung:}
\begin{align}
\Phi(\rho,\varphi) &= \begin{Bmatrix} 1 \\ \ln\rho \end{Bmatrix} \cdot \begin{Bmatrix} 1 \\ \varphi \end{Bmatrix} + \begin{Bmatrix} \rho^n \\ \rho^{-n} \end{Bmatrix} \cdot \begin{Bmatrix} \cos(n\varphi) \\ \sin(n\varphi) \end{Bmatrix}
\end{align}



\subsection{Magnetostatik}

\textbf{Hochpermeable Materialien:}\\
Für $\mu \gg \mu_0$ wird die magnetische Feldstärke stark reduziert, da das Material die magnetischen Feldlinien effizient führt.

\begin{align}
\mu \rightarrow \infty \quad \Rightarrow \quad \vec{H} \rightarrow 0 \quad \text{(bei endlichem } \vec{B}\text{)}
\end{align}




\textbf{Biot-Savart:}\\
Magnetfeld $\vec{H}(\vec{r}_p)$ erzeugt durch stromdurchflossenen Leiter.

\begin{align}
\vec{H}(\vec{r}_p) = \frac{I}{4\pi} \int\limits_C \frac{d\vec{l} \times \vec{r}_d}{|\vec{r}_d|^3}
\end{align}


Aufbau des Magnetfeld-Problems:
\begin{itemize}
\item $\vec{r}_p$ -- Ortsvektor zum Aufpunkt P (wo wir das Magnetfeld berechnen wollen)
\item $\vec{r}_q \text{ oder }\vec{r}'$ -- Ortsvektor zum Quellpunkt Q (Punkt auf dem stromdurchflossenen Leiter)
\item $d\vec{l} = d\vec{r}_q$ -- Leiterelement, infinitesimales Vektorelement entlang des Leiters, zeigt in Richtung des Stroms
\item $\vec{r}_d = \vec{r}_p - \vec{r}_q$ -- Abstandsvektor vom Quellpunkt Q zum Aufpunkt P
\item $d\vec{l} \times \vec{r}_d$ -- Kreuzprodukt: Bestimmt die Richtung des Magnetfeldbeitrags (senkrecht zu beiden Vektoren)
\item $|\vec{r}_d|^3$ -- Abstand im Nenner: Sorgt für die $1/r^2$-Abhängigkeit des Magnetfelds (mal ein zusätzlicher Faktor $r$ aus der Vektornormierung)
\item $C$ -- Integrationspfad entlang des stromdurchflossenen Leiters
\end{itemize}


Beispiel Segment: Von $(a,0,0)$ bis $(a,a,0)$:

\begin{align}
d\vec{r}_q(t) &= (a, t, 0) \quad \text{mit } t \in [0,a] \text{ parametrisiert } C \\
d\vec{l} &= \frac{d\vec{r}_q}{dt} \, dt
\end{align}







\textbf{Gegeninduktivität:}\\
Die Gegeninduktivität $M$ beschreibt die magnetische Kopplung zwischen zwei Stromschleifen.

\begin{align}
M = \frac{\Psi}{I}
\end{align}

Dabei ist:
\begin{itemize}
\item Induzierte Spannung: $U_2 = -M \frac{dI_1}{dt}$ (Faraday'sches Induktionsgesetz)
\end{itemize}

\section{Magnetoquasistik MQS}

Vereinfachte Maxwell-Gleichungen ($\frac{\partial \vec{D}}{\partial t} \ll \vec{J}$):

\begin{itemize}
\item $\text{rot } \underline{\vec{E}} + j\omega \underline{\vec{B}} = 0$
\item $\text{rot } \underline{\vec{H}} = \underline{\vec{J}}$
\end{itemize}

Bedingung: $\frac{\omega \varepsilon}{\kappa} \ll 1$\\
$\rightarrow$ Herleitung: $j\omega \underline{\vec{D}} \ll \underline{\vec{J}}$





\textbf{Diffusionsgleichung:} (MQS; $\kappa$, $\mu$ konst.; $\text{div } \vec{E} = \text{div } \vec{H} = 0$) \hfill $\rightarrow$ Herleitung: $\text{rot } \vec{E} + \frac{\partial \vec{B}}{\partial t} = 0$ und
\begin{flushright}
\phantom{$\rightarrow$ Herleitung: }$\text{rot } \vec{H} - \frac{\partial \vec{D}}{\partial t} = \vec{J}$
\end{flushright}


Zeitbereich, z.B. 1D-Funktion $H(x,t)$:
\begin{equation}
\frac{\partial^2 H}{\partial x^2} - \kappa\mu \frac{\partial H}{\partial t} = 0
\end{equation}


Fourier-Transformation (Frequenzbereich):
\begin{align}
\frac{d^2 \underline{H}}{dx^2} - \underline{p}^2 \underline{H} &= 0 \quad \text{mit } \underline{p}^2 = j\omega\kappa\mu \\
\Delta \underline{\vec{H}} - \underline{p}^2 \underline{\vec{H}} &= 0 \\
\Delta \underline{\vec{E}} - \underline{p}^2 \underline{\vec{E}} &= 0
\end{align}


Laplace-Transformation $\underline{H}(x, \underline{s})$ (Anfangsbedingung $H(x, t = 0) = 0$):

\begin{equation}
\frac{\partial^2 \underline{H}(x,\underline{s})}{\partial x^2} - \underline{p}^2 \underline{H}(x,\underline{s}) = 0 \quad \text{mit } \underline{p}^2 = \kappa\mu \underline{s}
\end{equation}

Allgemeine Lösung (für Fourier- und Laplace-Transformation mit Anfangsbedingung $H(x, t = 0) = 0$):

$\rightarrow$ siehe \hyperlink{helmholtz_allgemeine_loesung}{allgemeine Helmholtz-Gleichung}

\textbf{Feldabfall (bei MQS):}


Exponentieller Abfall im quasistatischen Fall:
\begin{align}
|\underline{H}(d)| = H_0 e^{-\frac{d}{\delta}}
\end{align}


Feldverhältnis der Ränder: 
\begin{align}
\frac{H(d)}{H(0)}
\end{align}


\textbf{Poyntingscher Satz:}
\begin{align}
-\frac{\partial w_{em}}{\partial t} = \text{div}(\vec{S}) + p_V
\end{align}

\section{Wellen}

\textbf{Ausbreitung:}\\
Welle mit Ausbreitungsrichtung $\vec{e}_k = \vec{e}_z$. Daher nur $E_x$ und $H_y$ Komponente.

\begin{align}
\vec{S} = \vec{E} \times \vec{H} = \begin{pmatrix} E_x \\ 0 \\ 0 \end{pmatrix} \times \begin{pmatrix} 0 \\ H_y \\ 0 \end{pmatrix} = \begin{pmatrix} 0 \\ 0 \\ E_x H_y \end{pmatrix}
\end{align}


\textbf{Komplexe Felddarstellung:}
\begin{align}
H(z,t) &= \text{Re}\{\underline{H}(z) \cdot e^{j\omega t}\} \\
&= \text{Re}\{\underline{H}_0 e^{-z/\delta} e^{-jz/\delta} e^{j\omega t}\} \\
&= \text{Re}\{\underline{H}_0 e^{-z/\delta} e^{j(\omega t - z/\delta)}\} \\
&= H_0 e^{-z/\delta} \cos\left(\omega t - \frac{z}{\delta}\right)
\end{align}


\textbf{Wellengleichung:}($\epsilon$, $\mu$ konst.; $\kappa = 0$; $\vec{J} = 0$; $\text{div } \vec{E} = \text{div } \vec{H} = 0$) \hfill $\rightarrow$ Herleitung: $\text{rot } \vec{E} + \frac{\partial \vec{B}}{\partial t} = 0$ und
\begin{flushright}
\phantom{$\rightarrow$ Herleitung: }$\text{rot } \vec{H} - \frac{\partial \vec{D}}{\partial t} = \vec{J}$
\end{flushright}
Für eine 1D-Funktion $f(x,t)$
\begin{equation}
\frac{\partial^2 f}{\partial x^2} - \epsilon\mu \frac{\partial^2 f}{\partial t^2} = 0
\end{equation}

Fourier-Transformation (Frequenzbereich):
\begin{align}
\Delta \underline{\vec{E}} + k^2 \underline{\vec{E}} &= 0 \quad \text{mit } k^2 = \omega^2\epsilon\mu\\
\Delta \underline{\vec{H}} + k^2 \underline{\vec{H}} &= 0
\end{align}


D'Alembert'sche Lösung:
\begin{equation}
f(x, t) = F(x - ct) + G(x + ct), \quad c = \frac{1}{\sqrt{\epsilon\mu}}
\end{equation}

Allgemeine Lösung:

$\rightarrow$ siehe \hyperlink{helmholtz_allgemeine_loesung}{allgemeine Helmholtz-Gleichung}














\textbf{Komplexe Permittivität (verlustbehaftete Medien):}\\
In verlustbehafteten Dielektrika wird die Permittivität komplex:

\begin{align}
\underline{\varepsilon} &= \varepsilon' - j\varepsilon'' = \varepsilon_0(\varepsilon_r' - j\varepsilon_r'') \\
\underline{\varepsilon} &= \varepsilon'(1 - j\tan\delta)
\end{align}

\begin{itemize}
    \item $\varepsilon'$: Realteil (Energiespeicherung, Polarisation)
    \item $\varepsilon''$: Imaginärteil (Verluste, Dissipation)
\end{itemize}

\textbf{Verlustwinkel (nicht die Eindringtiefe):}
\begin{equation}
\tan(\delta) = \frac{\varepsilon''}{\varepsilon'}
\end{equation}

Für kleine Verluste gilt: $\delta \ll 1$ (guter Isolator)



\textbf{Wellenimpedanz}

\begin{itemize}
\item $Z_0 = \sqrt{\frac{\mu_0}{\varepsilon_0}}$ -- Wellenimpedanz im Vakuum (Freiraumimpedanz) \hfill $[Z_0] = \Omega$
\item $Z = \frac{E_x}{H_y} = \sqrt{\frac{\mu}{\varepsilon}}$ -- Verhältnis von E- zu H-Feld (Wellenimpedanz) \hfill $[Z] = \Omega$
\item $\vec{H} = \frac{1}{Z} \vec{k} \times \vec{E}$ bzw. $H_y = \frac{E_x}{Z}$ -- E-H-Beziehung für ebene Wellen \hfill $[H] = \frac{\text{A}}{\text{m}}$
\item $Z = \sqrt{\frac{\mu}{\varepsilon}} \approx \sqrt{\frac{\mu}{\varepsilon'}}$ -- Wellenimpedanz im verlustbehafteten Medium (reell für kleine Verluste) \hfill $[Z] = \Omega$
\end{itemize}


\textbf{Äquivalente Leitfähigkeit:}
\begin{equation}
\kappa = \omega\varepsilon'' = \omega\varepsilon'\tan(\delta)
\end{equation}

\begin{equation}
\varepsilon'' = \frac{\kappa}{\omega}
\end{equation}

\textbf{Komplexe Wellenzahl:}
\begin{equation}
k = \beta - j\alpha = \omega\sqrt{\mu\varepsilon_c}
\end{equation}

\begin{itemize}
    \item $\beta$: Phasenkonstante (bestimmt Wellenlänge)
    \item $\alpha$: Dämpfungskonstante (bestimmt Absorption)
\end{itemize}


Für kleine Verluste ($\tan \delta \ll 1$):

\begin{equation}
k = \omega\sqrt{\mu_0 \varepsilon} = \omega\sqrt{\mu_0 \varepsilon'(1 - j\tan\delta)} \approx \beta - j\alpha
\end{equation}

\begin{align}
\beta &= \omega\sqrt{\mu_0 \varepsilon'} = \omega\sqrt{\mu_0 \varepsilon_r \varepsilon_0} = \frac{\omega}{c}\sqrt{\varepsilon_r}\\
\alpha &= \frac{\beta \tan \delta}{2} = \frac{\omega\sqrt{\mu_0 \varepsilon'} \tan \delta}{2}
\end{align}


Für kleine Verluste ($(1 + p)^q \approx 1 + pq \quad \text{für } p = \tan \delta \ll 1$):

\begin{equation}
\sqrt{1 - j\tan\delta} \approx 1 - j\frac{\tan\delta}{2}
\end{equation}




\textbf{Gedämpfte Wellenausbreitung:}
\begin{equation}
\underline{E}(z) = E_0 e^{-j k z} = E_0 e^{-\alpha z} e^{-j\beta z}
\end{equation}

Leistungsflussdichte nimmt exponentiell ab:
\begin{equation}
S_z(z) = S_0 e^{-2\alpha z}
\end{equation}

\textbf{Dämpfungsstrecke} ($-3$ dB, d.h. Halbierung der Leistung):
\begin{equation}
d = \frac{\ln(2)}{2\alpha} = \frac{\ln(2)}{2\kappa}
\end{equation}






\textbf{Poyntingscher Satz}

\begin{align}
\text{div}\,\overline{\vec{S}} = -\overline{p_V}
\end{align}


Die linke Seite beschreibt die \textbf{Änderung der Leistungsflussdichte} entlang der Ausbreitungsrichtung, die rechte Seite die \textbf{dissipierte Leistung} im Medium.

differentielle Form im Frequenzbereich:

\begin{align}
\text{div}\,\underline{\vec{S}} = -\frac{1}{2}\underline{\vec{J}}^* \cdot \underline{\vec{E}} - j\omega\left(\overline{w_m} - \overline{w_e}\right)
\end{align}





Verknüpfung von $p$ und $k$:
\begin{equation}
p = jk
\end{equation}


\section{Klausur}

\begin{itemize}
\item Klausurdauer: 2,5 h
\item Keine Hilfsmittel: Nur Stift, kein Taschenrechner, kein Hilfszettel, keine Formelsammlung etc.
\item 2. Termin: 07.10.25, 8:30-11:00 HE101
\item Format: 1) kurze unabhängige Aufgaben (wie HA5)\\
2) 3) 4) Rechenaufgaben (drei von den 5 Themen zu rechnen - halbe Stunde pro Aufgabe)
\item Besselfunktion kommt nicht ran (nicht lernen)
\item Wellen: wahrscheinlich keine komplette Rechenaufgabe, aber ein, zwei Teilaufgaben in 1) - z.B. Polarisation/Dipol, gehört haben, aber nicht zum Rechnen $\rightarrow$ Wenn in der Aufgabe: ohne Rechnung $\rightarrow$ Dann auch wirklich nur Ergebnis.
\item Meiste kommt aus der Übung. Wenn die Übung alle verstanden haben, dann gut für Klausur.
\item Altklausuren, die 10 Jahre alt sind, sind noch einigermaßen okay aktuell (reicht zum Lernen).
\item konstaten auswendig lernen nicht notwendig (außer höchstens C). Z.B.: Lösung 4 $\epsilon_0$ reicht aus
\item Einheitscheck immer ganz gut
\item Maxwell wird immer abgefragt: wie 1. Aufgabe in der Problekalusur, oder auch in Frequenz, oder Integralform
\item Herleitung für Bedingung für MQS könnte abgefragt werden: $\frac{\omega \varepsilon}{\kappa} \ll 1$
\item rot H = dD/dt + J -> (d/dt zu jw)  - Trafo der Maxwell-Gleichung mit Fourier könnte auch abgefragt werden
\item Biot-Savart müssen wir nicht auswendig kennen für die Klausur, auch nicht herleiten können, aber anwenden können. Integrallösung üben muss nicht geübt werden. Nur aufschreiben, verstehen und anwenden.
\item div A = 0 -- Coulomb-Eichung


\end{itemize}
\end{document}
